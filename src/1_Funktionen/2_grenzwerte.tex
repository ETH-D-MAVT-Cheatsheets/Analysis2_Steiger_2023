\subsection{Grenzwerte}
    \subsubsection{Bernoulli de L'H$\hat{\textrm{o}}$pital}
        Falls $\displaystyle \lim_{x \to a} f(x) = \lim_{x \to a} g(x) = 0$ (oder $\pm \infty$), so gilt
        $$
            \lim_{x \to a} \frac{f(x)}{g(x)} = \lim_{x \to a} \frac{f'(x)}{g'(x)}.
        $$
    \subsubsection{Landau Symbol}
        \vspace{-0.5em}
        \begin{align*}
            \fbox{$f(x) = o(g(x)) \textrm{ für } x \to a$} &\Longleftrightarrow \lim_{x \to a} \frac{f(x)}{g(x)} = 0\\[0.5em]
            \fbox{$f(x) = O(g(x)) \textrm{ für } x \to a$}  &\Longleftrightarrow \lim_{x \to a} \left\lvert\frac{f(x)}{g(x)}\right\rvert \leq A \in \mathbb{R}
        \end{align*}
        $\textbf{Es gilt:}$
            \begin{align*}
                x^k &= o(e^x) \quad \textrm{ für } x \to \infty, \quad k \in \mathbb{R}\\
                \ln(x) &= o(x^k) \quad \textrm{ für } x \to \infty, \quad k > 0\\[0.5em]
                f(x) &= o(g(x)) \; \Rightarrow \; f(x) = O(g(x))\\
                f(x) &= O(g(x)) \; \nRightarrow \; f(x) = o(g(x))
            \end{align*}
    \subsection{Eigenschaften}
        Eine Funktion $f : A \to B$ ist eine Vorschrift, die jedem $x \in A$ ein Element $f(x) \in B$ zuordnet, $f: x \to f(x)$.
        \begin{description}
            \item[Definitionsbereich:] $D(f) = A$
            \item[Zielbereich:] $Z(f) = B$
            \item[Wertebereich:] $W(f) = \{ f(x) \vert \ x \in D(f)\}$   
        \end{description}
        \subsubsection{Surjektiv}
            Jeder Wert im Zielbereich $Z(f)$ wird angenommen.
            \mathbox{
                W(f) = Z(f)
            }
        \subsubsection{Injektiv}
            Jede Horizontale schneidet den Graphen $\Gamma(f)$ höchstens einmal.
            \begin{itemize}
                \item $f(x_1) = f(x_2) \Rightarrow x_1 = x_2$, sonst nicht injektiv
            \end{itemize}
        \subsubsection{Bijektiv}
            \begin{center}
                Injektiv \& Surjektiv $\Leftrightarrow$ Bijektiv $\Leftrightarrow$ Umkehrbar
            \end{center}
        \subsubsection{Inverse Funktion}
            Sei $f(x)$ eine Funktion von $D(f)$ nach $W(f)$, dann ist $f^{-1}: W(f) \to D(f)$ mit $y \mapsto f^{-1}(y)$ die inverse Funktion von $f(x)$.
            \begin{itemize}
                \item $W(f^{-1}) = D(f)$
                \item $D(f^{-1}) = W(f)$
            \end{itemize}
        \subsubsection{Gerade \& Ungerade}
            \begin{description}
                \item[gerade:]\phantom{as} $f(-x) = f(x)$ 
                \item[ungerade:] $f(-x) = -f(x)$ 
            \end{description}
        \subsubsection{Stetigkeit}
            $f(x)$ ist stetig im Punkt $\xi$ falls
            $$
                \lim_{x\to\xi^-} f(x) = f(\xi) = \lim_{x\to\xi^+} f(x).
            $$
            \begin{itemize}
                \item Bei Lücken in $D(f)$ werden die einzelnen Abschnitte separat betrachtet.
            \end{itemize}
        \subsubsection{Monotonie}
            \textbf{(Strikt) Monoton Steigend}
                \begin{itemize}
                    \item $x_1 < x_2\ \Longleftrightarrow\ f(x_1) \leq f(x_2)$ \hfill (strikt: $<$)
                    \item $f'(x) \geq 0$ \hfill (strikt: $>$)
                \end{itemize}
            \textbf{(Strikt) Monoton Fallend}
                \begin{itemize}
                    \item $x_1 < x_2\ \Longleftrightarrow\ f(x_1) \geq f(x_2)$ \hfill (strikt: $>$)
                    \item $f'(x) \leq 0$ \hfill (strikt: $<$)
                \end{itemize}
        \subsubsection{Beschränktheit}
            Alle Funktionswerte sind in einem endlich breiten waagerechten Parallelstreifen enthalten.