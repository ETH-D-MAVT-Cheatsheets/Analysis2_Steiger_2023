% !TeX root = ../../ZF_bmicha_Ana.tex
\subsection{Tangentialebenen}
    \subsubsection{Linearisierungsformel}
        \mathbox{
            z = f(x_o,y_o) + f_x(x_o,y_o) (x\!-\!x_o) + f_y(x_o,y_o)(y\!-\!y_o)
        }
        \mathbox{
            0 = f_x(x_o,y_o,z_o)(x\!-\!x_o) + f_y(x_o,y_o,z_o)(y\!-\!y_o) + \dots
        }
    \subsubsection{Gradient}
        \begin{itemize}
            \item $f(x,y,z) = C$ ist eine Niveaufläche
            \item $\grad(f)$ steht senkrecht auf Niveauflächen. ($\to \vec{n}$ )
            \item Ebene mit Normalenvektor $\vec{n}= (A,B,C)^T$:
            $$
                Ax + By + Cz = D
            $$
        \end{itemize}