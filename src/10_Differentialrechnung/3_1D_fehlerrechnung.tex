% !TeX root = ../../ZF_bmicha_Ana.tex
\subsection{Fehlerrechnung}
    Die berechnete Grösse $f$ ist abhängig von der gemessenen Grösse $x$.
    Die gemessene Grösse weicht mit dem Messfehler $dx$ von der Realität ab.
    \begin{itemize}
        \item \textbf{Linearisierung}
            \vspace*{-0.5em}
            \mathbox{
                f(x) \approx  f(x_0) + \left. \frac{\partial f}{\partial x} \right|_{x_0} \cdot (x - x_0)
            }
        \item \textbf{Absoluter Fehler}
            \vspace*{-0.5em}
            \mathbox{
                \Delta f = f(x + \Delta x) -\! f(x) \quad \overset{\Delta x \to 0}{\longrightarrow} \quad \Delta f \approx f'(x)\ \Delta x
            }
        \item \textbf{Relativer Fehler}
            \vspace*{-0.5em}
            \mathbox{
                \frac{df}{f}
            }
    \end{itemize}
    \subsubsection{Bemerkungen}
        \vspace{0.5em}
        \begin{minipage}{0.54\linewidth}
            \centering \vspace{4pt}
            $1\%$ Genauigkeit
            $$
                \frac{\Delta x}{x} = 1\% = \frac{1}{100}
            $$          
        \end{minipage}
        \begin{minipage}{0.45\linewidth}
            \centering
            Messfehler von $1^\circ$
            $$
                \Delta \alpha = \frac{\pi}{180}
            $$
        \end{minipage}