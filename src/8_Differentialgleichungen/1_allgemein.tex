% !TeX root = ../../ZF_bmicha_Ana.tex
\subsection{Eigenschaften}
    \subsubsection{Existenz- \& Eindeutigkeitssatz}
        Sei $y' = f(x,y)$ in $D(f)$ \textbf{stetig} und \textbf{stetig partiell nach $\boldsymbol{y}$ differenzierbar}.
        \begin{itemize}
            \item[$\Rightarrow$] Für jeden Punkt $(x_o,y_o) \in D(f)$ gibt es \textit{genau eine} Lösung. 
            \item[$\Rightarrow$] Graphen von versch. Anfangswertproblemen (AWP) sind identisch oder disjunkt (kein gem. Pkt.).
        \end{itemize}
    \subsubsection{Linear}
        Eine DGL heisst \textit{linear}, falls alle $y$-Terme ($y, y', y'', \dots$) nur linear vorkommen.
        ($\to$ Kein: $sin(y')$, $y\cdot y'$, $e^y$, \dots)

% \textbf{Nützliche Integrale}
%     \begin{align*}
%         \int \frac{a}{ax + b} \ dx &= \ln(ax + b) + C, \quad C \in \mathbb{R}\\
%         \int \frac{\frac{1}{a}}{1 + \left(\frac{x}{a}\right)^2} \ dx &= \arctan\left(\frac{x}{a}\right) + C, \quad C \in \mathbb{R}
%     \end{align*}
% \textbf{Ordnung:}\\
%     Die Ordnung einer DGL entspricht dem Grad der höchsten Ableitung.