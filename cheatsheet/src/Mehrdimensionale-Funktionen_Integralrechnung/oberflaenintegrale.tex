% !TeX root = ../../ZF_bmicha_Ana.tex
\subsection{Oberflächenintegrale}
    Spezialfall eines Flächenintegrals.\\    
    Flächeninhalt einer parametr. Oberfläche berechnen:
    \mathbox{
        \iint_{\mathcal{O}} d\mathcal{O} = \iint_{\mathcal{O}} \abs*{\vec{r}_u \times \vec{r}_v} du dv
    }
    \textbf{Parametrisierung der Oberfläche:}
    $$
        \vec{r} = \begin{pmatrix}
            x(u,v)\\
            y(u,v)\\
            z(u,v)
        \end{pmatrix}
    $$
    \begin{description}
        \item[Oberflächenelement:] $d\mathcal{O} = \abs*{\vec{r}_u \times \vec{r}_v} du dv$
        \item[Normalenvektor:]  $\vec{n} = \vec{r}_u \times \vec{r}_v$
        \item[Normaleneinheitsvektor:]  $\vec{n}_o = \frac{\vec{r}_u \times \vec{r}_v}{\abs*{\vec{r}_u \times \vec{r}_v}}$
    \end{description}
    {\small Oberflächenelement entspricht Jacobi-Determinante.}