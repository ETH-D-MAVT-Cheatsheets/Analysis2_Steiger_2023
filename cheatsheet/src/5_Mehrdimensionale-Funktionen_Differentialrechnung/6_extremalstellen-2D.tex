% !TeX root = ../../ZF_bmicha_Ana.tex
\subsection{Extremalstellen von $f(x,y)$}
    \vspace{0.25em}
    \begin{enumerate}
        \item Inneres untersuchen $\to$ $\textrm{grad}f \overset{!}{=} 0$
        \item Rand untersuchen
        \begin{itemize}
            \item Lagrange Multiplikatoren
            \begin{enumerate}
                \item $g(x,y)$ beschreibt Rand
                \item $\textrm{grad}f(x_o,y_o) = \lambda \cdot \textrm{grad}\, g(x_o,y_o)$\\[0.25em]
                      \phantom{llll}$\textrm{grad}\, g(x_o,y_o) \neq 0,\phantom{ll} \lambda \in \mathbb{R}$
                \item Gleichungssystem aus (a) und (b) lösen.
            \end{enumerate}
            \item Parametrisierung
            \begin{enumerate}
                \item Rand parametrisieren
                \item Parametrisierung in $f$ einsetzen
                \item Nach Parameter ableiten und nullsetzen.\\[0.25em] \phantom{llll}$f'(t) = 0$
            \end{enumerate}
        \end{itemize}
        \item Eckpunkte untersuchen
        \item Kandidaten vergleichen
        \item Art der Extremalstelle: Hesse-Matrix aufstellen $\mathcal{H} = 
        \left[
            \begin{array}{c c} 
                \frac{\partial^2 f}{\partial x^2}       & \frac{\partial^2 f}{\partial x \partial y}\\
                \frac{\partial^2 f}{\partial x \partial y}  & \frac{\partial^2 f}{\partial y^2}
            \end{array}
        \right]$
            \begin{itemize}
                \item Maximum: $\mathcal{H}$ ist negativ definit ($\lambda_i < 0$)
                \item Minimum: $\mathcal{H}$ ist positiv definit ($\lambda_i > 0$)
                \item Sattelstelle: $\mathcal{H}$ ist indefinit ($\mathcal{H}$ besitzt $\lambda_i > 0$ \textbf{und} $\lambda_i < 0$)
            \end{itemize}
    \end{enumerate}