% !TeX root = ../../ZF_bmicha_Ana.tex
\subsection{Schwerpunkt / Trägheitsmoment}
    Sei $H(x)$ die Höhe des Fläche a.d.S. $x$.\\
    Sei $\sigma$ die Flächendichte $[kg/m^2]$.
    \begin{align*}
        \textrm{Fläche: }  A = \int_{x_1}^{x_2} H(x)\ dx\\
        \textrm{Masse: }  M = \int_{x_1}^{x_2} \sigma \cdot H(x)\ dx\\
        \textrm{Schwerpunkt: }  x_s = \frac{1}{M} \int_{x_1}^{x_2} x \cdot \sigma \cdot H(x)\ dx\\
        \textrm{SP Rotationsvolumen: } x_s = \frac{1}{V} \int_{x_1}^{x_2} x \cdot \pi \cdot H^2(x)\ dx\\
        \textrm{Trägheitsmoment: }  I_y = \int_{x_1}^{x_2} x^2 \cdot \sigma \cdot H(x)\ dx
    \end{align*}
    \subsubsection{Trägheitsmoment}
    \vspace*{-1em}
        \begin{align*}
            \Theta =& \int (\textrm{Abstand zur Rotationsachse})^2 \cdot (\textrm{Masse})\\
            \Theta =& \; \rho \cdot \int_a^b x^2 \cdot G(x) dx\\
            J_0 =& \; \frac{\pi R^4}{2} = \; \parbox{5cm}{polares Flächenträgheitsmoment\\ der Kreisscheibe}\\
            \Theta_x =& \; \rho \cdot \int_a^b \frac{1}{2} \pi (f(x))^4 dx = \; \parbox{5cm}{Masseträgheitsmoment\\ eines Rotationskörpers\\ um die x-Achse}\\
            \Theta =& \; \rho \cdot \frac{1}{2} \pi \int_a^{b} y(t)^4\|\dot{x}(t)\| dt\\
            G(x) =& \; \text{Masse an diesem Abstand}\\
            M(x) =& \; \text{Mantelfäche} = 2 \pi x \cdot G(x) = \text{Umfang} \cdot \text{Höhe}\\ %\underbrace{2 \pi x \vphantom{G(x)}}_{\text{Umfang}} \cdot \underbrace{G(x)}_{\text{Höhe}}\\
            \Theta_z =& \; \rho \int_{x_1}^{x_2} x^2 \cdot M(x) dx
        \end{align*}        