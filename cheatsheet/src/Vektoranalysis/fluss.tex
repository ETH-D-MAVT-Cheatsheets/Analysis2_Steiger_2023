% !TeX root = ../../ZF_bmicha_Ana.tex
\subsection{Fluss \hfill $\Phi$}
    \vspace{-1em}
    \begin{align*}
        \Phi &= \iint_A \vec{v} \cdot \vec{n_0} \ dA &\textrm{Allgemein}\\
        \Phi &= \iint_A \vec{v}(\vec{r}(u,v)) \cdot (\vec{r}_u \times \vec{r}_v) \ du dv &\textrm{Parametr.}\\
        \Phi &= \iiint_V \div(\vec{v}) \ dV &\textbf{Satz\ v. Gauss}
    \end{align*}

    \subsubsection{Satz von Gauss}
        Falls $\vec{v}$ in ganz $B$ \textbf{definiert} und einmal \textbf{stetig differenzierbar} (\textit{regulär}) ist, gilt
        $$
            \iint_{\partial B} \vec{v} \cdot \vec{n_0} \ dO = \iiint_B \div(\vec{v}) \ dV,
        $$
        wobei $\partial B$ die geschlossene Oberfläche des Volumens $B$ bezeichnet.
        Der Normaleneinheitsvektor $\vec{n_0}$ auf $\partial B$ zeigt von innen nach aussen.